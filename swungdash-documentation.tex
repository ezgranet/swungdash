\documentclass[12pt]{article}
\usepackage{swungdash}
\usepackage{xcolor}
\usepackage{fontspec}\definecolor{darkspringgreen}{rgb}{0.09, 0.45, 0.27}
\usepackage{titlesec}
\titleformat{\subsection}
  {\bfseries}{\thesection.\thesubsection}{1em}{\normalfont\bfseries}
\usepackage[hidelinks]{hyperref}
\usepackage{hologo}
\usepackage[british]{babel}
\usepackage[useregional]{datetime2}
\DTMlangsetup[en-GB]{ord=omit}
\definecolor{LightGray}{gray}{0.9}
%\usepackage{mathpazo}
\IfFontExistsTF{Palatine Parliamentary}{\setromanfont[RawFeature={+calt,+hlig,+liga,+dlig,+onum,+pnum}]{Palatine Parliamentary}
}{\setromanfont[RawFeature={+onum,+pnum}]{TeX Gyre PagellaX}}
\setmonofont[Scale=.9,BoldFont=Source Code Pro Bold]{Source Code Pro}

\usepackage{minted}
\date{\today\\\smallskip\ttfamily Version \swungdashversionnumber}
\author{Elijah Z Granet\thanks{e-mail: \href{mailto:me@ezgra.net}{\ttfamily me@ezgra.net}}}

\title{\texttt{swungdash}:\\A package for a swung dash}
\begin{document}
\maketitle
\tableofcontents
\clearpage
\section{Overview}
The `swung dash' (\swungdash) is a rare but very useful mark of punctuation used in typesetting dictionaries and reference works as a stand in to avoid repeating the defined term in examples or definitions, and thus to save space.  For example, a quoted example of the term `extraterritoriality' might save space by omitting repeating the long word:
\begin{quote}
	\swungdash, in this as in every other case, is a fiction only, for diplomatic envoys are in reality not without, but within, the territories of the receiving States. The term `\swungdash' is nevertheless valuable because it demonstrates clearly the fact that envoys must, in most respects, be treated as though they were not within the territory of the receiving States. \footnote{Sir H  Lauterpacht \textsc{qc} (ed) \textit{Oppenheim on International Law}, vol 1 (8\textsuperscript{th} edn, Longmans, Green \& C\textsuperscript{o} 1955),  793}
\end{quote}

Although the swung dash is included in Unicode  as \texttt{U+2053}, few typefaces include it.  This package turns a tilde in any given typeface into a swung dash of 1 \textsc{m} width, and then, using the \texttt{accsup} package, overwrites this extended tilde into the Unicode character so that it will be read as a swung dash.
\section{Usage}
In your preamble put:
\begin{minted}[
framesep=2mm,
baselinestretch=1.2,
bgcolor=LightGray,
fontsize=\footnotesize,
breaklines,
firstnumber=last
]
{latex}
\usepackage{swungdash}
\end{minted}

To typeset a swung dash, simply use the command:
\begin{minted}[
framesep=2mm,
baselinestretch=1.2,
bgcolor=LightGray,
fontsize=\footnotesize,
breaklines,
firstnumber=last
]
{latex}
\swungdash
\end{minted}
\section{Development}
Bugs, feature requests, \textit{etc}, should be submitted to the project's official Githup page: (\url{github.com/ezgranet/swungdash}).
\section{Implementation}
\begin{minted}[
frame=lines,
framesep=2mm,
baselinestretch=1.2,
bgcolor=LightGray,
fontsize=\footnotesize,
linenos,
breaklines,
firstnumber=last
]
{latex}

\def\swungdashversionnumber{1.0.0}
\ProvidesPackage{swungdash}
  [2022/08/24 v1.0 code to typeset a swung dash]
    % !TeX program = lualatex                                   
% !TeX encoding = utf8
% This work may be distributed and/or modified under the 
% conditions of the LaTeX Project Public License, either version 1.3 
% of this license or (at your option) any later version.
% The latest version of this license is in
%   http://www.latex-project.org/lppl.txt
% and version 1.3 or later is part of all distributions of LaTeX 
% version 2005/12/01 or later.
%
% This work has the LPPL maintenance status `maintained'.
%
% The Current Maintainer of this work is Elijah Z Granet

\RequirePackage{accsupp}\RequirePackage{graphicx}
\RequirePackage{iftex}
\newcommand{\thetilde}{\symbol{"007E}}
\newcommand{\swungdash}{\BeginAccSupp{method=hex,unicode,ActualText= 2053}%
\resizebox{1em}{!}{{\raisebox{-.1ex}{\scalebox{1.75}[1.1]{\thetilde}}}}
\EndAccSupp{}%
}
\newcommand{\twiddle}{$\sim$}
\ifPDFTeX   {
    \PackageError{swungdash}
      {You are using pdfTeX but this package only works 
      \MessageBreak with XeTeX or LuaTeX}{}
    }
\fi

\end{minted}

\section{Version History}
\subsection{\texttt{1.0.0}}

\ttfamily 24 August 2022: Package creation

	
\end{document}